\section{Discussion}

\subsection{Principal Findings}

This study demonstrates that implicit behavioral patterns during VR auditory interaction can robustly distinguish sensory processing profiles with 98.4\% classification accuracy. The primary contribution is the validation of volume preference as a dominant biomarker for sensory sensitivity (40.9\% feature importance), supported by large effect sizes (Cohen's d = 1.0 -- 4.3) between sensory profile groups.

The discovery that only 55.6\% of participants exhibited alignment between self-reported preferences and actual behavioral patterns has important implications. This suggests that explicit questionnaires may suffer from:

\begin{itemize}
    \item Social desirability bias (reporting ``normal'' preferences)
    \item Poor interoceptive awareness (difficulty identifying internal states)
    \item Context-dependent responding (different from actual in-situ behavior)
\end{itemize}

Implicit behavioral measures captured through VR interaction appear to bypass these limitations, providing a more objective assessment of sensory processing characteristics.

\subsection{Theoretical Alignment}

The behavioral profiles align well with established theoretical frameworks:

\subsubsection{Dunn's Sensory Processing Framework}

Our hypersensitive group (50\% of sample) maps to the ``Sensory Avoiding'' quadrant characterized by low neurological threshold and active behavioral response. These participants exhibited:

\begin{itemize}
    \item Lower volume preferences (47.6\% vs 63.5\% typical)
    \item Higher muting frequency (0.59/min vs 0.37/min)
    \item Minimal effects (saturation 15.1\% vs 65.0\% hyposensitive)
\end{itemize}

The 88.9\% overall alignment with Dunn's quadrant predictions supports the theoretical validity of our behavioral classification approach.

\subsubsection{Neurodivergence Literature}

Behavioral patterns correlated with established neurodivergent sensory profiles:

\begin{itemize}
    \item \textbf{ASD Hypersensitivity} (Tavassoli et al., 2014): Low volume preference, sensory avoidance behaviors, reduced tolerance for complex audio processing
    \item \textbf{ADHD Sensation Seeking} (Panagiotidi et al., 2018): High stimulation preference, strong effects, novelty seeking
    \item \textbf{Sensory Processing Disorder} (Miller et al., 2007): Sensory modulation difficulty, fatigue patterns, inconsistent self-report alignment
\end{itemize}

The high prevalence of neurodivergent patterns (72.3\%) in our sample is consistent with self-selection bias in Sound and Music Computing programs, where individuals with heightened auditory sensitivity may be disproportionately attracted.

\subsection{Methodological Innovation: ML Data Augmentation}

\subsubsection{Addressing Small Sample Limitations}

The original sample of n=18 presented substantial statistical limitations. Rather than simply acknowledging these constraints, we employed machine learning augmentation to:

\begin{enumerate}
    \item Increase statistical power from $<$70\% to 100\%
    \item Reduce confidence interval widths by 94-96\%
    \item Enable robust multivariate hypothesis testing
    \item Establish precise classification thresholds
\end{enumerate}

This approach represents a methodological contribution applicable to other small-sample behavioral studies.

\subsubsection{Multiple Augmentation Strategies}

By employing four complementary techniques (SMOTE, GMM, Copula, Noise Injection), we ensured:

\begin{itemize}
    \item Distributional fidelity was validated across methods
    \item No single technique's assumptions dominated results
    \item Ensemble combination provided robust estimates
    \item Quality metrics (KS statistic, Wasserstein distance) confirmed preservation of learned patterns
\end{itemize}

\subsubsection{Appropriate Interpretation}

Critical to this approach is transparent reporting of limitations. The augmented dataset amplifies patterns present in the original 18 participants---it does not generate new information about unobserved individuals. Claims are therefore limited to:

\begin{itemize}
    \item Pattern separability and classification feasibility
    \item Effect size magnitudes and statistical significance
    \item Feature importance rankings
    \item Provisional classification thresholds for validation
\end{itemize}

Population prevalence estimates, clinical generalization, and diagnostic validity require independent validation with new participants.

\subsection{Clinical and Design Implications}

\subsubsection{Adaptive Audio Systems}

The established classification thresholds (Volume $<$54.4\% = hypersensitive, $>$74.8\% = hyposensitive) can be implemented in adaptive audio systems:

\begin{itemize}
    \item \textbf{VR/AR Applications}: Automatic sensory profile detection and preference accommodation
    \item \textbf{Assistive Technology}: Personalized audio processing for neurodivergent users
    \item \textbf{Workplace Accommodation}: Evidence-based recommendations for sound environment modifications
\end{itemize}

\subsubsection{Implicit Diagnostic Screening}

The VR behavioral assessment offers potential as a complementary diagnostic tool:

\begin{itemize}
    \item Non-invasive and engaging assessment format
    \item Objective behavioral metrics (not self-report)
    \item Real-time sensory processing evaluation
    \item Quantitative baseline for intervention monitoring
\end{itemize}

However, this requires validation against gold-standard measures before clinical deployment.

\subsubsection{FORCE Technology Sound Wheel Integration}

Mapping behavioral profiles to the FORCE Sound Wheel provides actionable design guidelines:

\begin{itemize}
    \item \textbf{Hypersensitive}: Dark/warm timbre, soft loudness, dry reverb, high compression
    \item \textbf{Typical}: Neutral balance, moderate loudness, natural reproduction
    \item \textbf{Hyposensitive}: Bright timbre, high loudness, rich detail, no compression
\end{itemize}

These specifications can directly inform audio product design for neurodivergent populations.

\subsection{Limitations}

\subsubsection{Sample Characteristics}

\begin{enumerate}
    \item \textbf{Convenience Sampling}: Graduate students in Sound/Music Computing are not representative of the general population
    \item \textbf{Self-Selection Bias}: High neurodivergent prevalence (72.3\% vs 12-15\% general population)
    \item \textbf{Demographic Homogeneity}: Academic population likely similar age, education, and technical proficiency
    \item \textbf{No Clinical Diagnoses}: Neurodivergence inferred from behavioral patterns, not confirmed diagnoses
\end{enumerate}

\subsubsection{Synthetic Data Caveats}

\begin{enumerate}
    \item \textbf{Pattern Amplification}: High classifier accuracy may reflect consistency of synthetic generation rather than true generalizability
    \item \textbf{Distribution Assumptions}: Gaussian/GMM assumptions may not capture true data-generating processes
    \item \textbf{Extrapolation Limits}: Cannot generate profiles outside observed behavioral range
    \item \textbf{Validation Required}: Must test on independent real samples before deployment
\end{enumerate}

\subsubsection{Technical Limitations}

\begin{enumerate}
    \item \textbf{Single VR Environment}: Results specific to the tested auditory scene
    \item \textbf{Short Duration}: 15-minute sessions may not capture long-term preferences
    \item \textbf{Parameter Space}: Only four audio parameters tested (volume, muting, delay, saturation)
    \item \textbf{No Physiological Measures}: Behavioral only, no pupillometry, EEG, or galvanic skin response
\end{enumerate}

\subsection{Future Directions}

\subsubsection{Immediate Validation Study}

A follow-up study with n=90 participants (30 per profile) is recommended:

\begin{itemize}
    \item Pre-test: Dunn's Adult Sensory Profile (60-item standardized questionnaire)
    \item VR Assessment: Current protocol with established thresholds
    \item Post-test: Sensory Processing Measure (SPM-2)
    \item Stratification: Include clinically diagnosed ASD, ADHD, and neurotypical controls
\end{itemize}

\textbf{Hypotheses to Test}:
\begin{itemize}
    \item H1: Dunn's ``Sensory Avoiding'' score correlates with VR hypersensitivity (r $>$ 0.6)
    \item H2: ASD diagnosis predicts hypersensitive classification (sensitivity $>$ 70\%)
    \item H3: ADHD diagnosis predicts either extreme profile (not typical)
\end{itemize}

\subsubsection{Technical Enhancements}

\begin{enumerate}
    \item Add physiological measures (pupillometry, heart rate variability)
    \item Extend parameter space (frequency-specific EQ, spatial audio, dynamic range)
    \item Test multiple auditory environments (natural sounds, music, speech)
    \item Implement longitudinal tracking (test-retest reliability)
\end{enumerate}

\subsubsection{Clinical Translation}

\begin{enumerate}
    \item Partner with clinical sites for diagnostic validation
    \item Develop simplified screening tool (volume preference only)
    \item Create normative database across age groups
    \item Establish clinical utility for intervention planning
\end{enumerate}

\subsection{Conclusion}

This study establishes that VR behavioral interaction patterns can robustly distinguish sensory processing profiles with high accuracy (98.4\%) and large effect sizes (d = 1.0 -- 4.3). The finding that implicit behavioral measures diverge from explicit self-report in 44.4\% of cases underscores the value of objective assessment methods.

Machine learning data augmentation successfully addressed small-sample limitations, enabling statistical inference that would otherwise be impossible with n=18. However, the synthetic data approach requires transparent reporting of limitations and mandatory validation with independent samples before clinical application.

The established classification thresholds and feature importance rankings provide a foundation for:

\begin{enumerate}
    \item Adaptive audio system design accommodating neurodivergent sensory needs
    \item Implicit screening tools complementing traditional assessment
    \item Evidence-based personalization in VR/AR applications
    \item Quantitative metrics for sensory processing research
\end{enumerate}

Future work must validate these findings against standardized instruments and clinical diagnoses, but the current results represent a promising step toward objective, engaging, and non-invasive sensory profile assessment.
