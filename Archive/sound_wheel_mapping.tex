\section{Results}

\subsection{Participant Classification Summary}

Based on the analysis of 20 participants' behavioral data and questionnaire responses, the following sensory profile distribution was observed:

\begin{itemize}
    \item \textbf{Hypersensitive}: 1 participant (5.0\%)
    \item \textbf{Hyposensitive}: 11 participants (55.0\%)
    \item \textbf{Typical}: 8 participants (40.0\%)
\end{itemize}

The predominance of hyposensitive classifications may be attributed to the convenience sampling method, which primarily recruited Sound and Music Computing students with high auditory engagement and stimulation-seeking tendencies.

\subsection{Behavioral Indicators by Sensory Profile}

Table \ref{tab:behavioral_summary} presents aggregated behavioral metrics for each sensory profile group.

\begin{table}[htbp]
\centering
\caption{Behavioral Metrics by Sensory Profile}
\label{tab:behavioral_summary}
\begin{tabular}{lccc}
\hline
\textbf{Metric} & \textbf{Hyper} & \textbf{Typical} & \textbf{Hypo} \\
\hline
Avg Volume (\%) & 51.6 & 54.2 & 59.3 \\
Mute Events & 4 & 5.9 & 8.1 \\
Total Adjustments & 249 & 516 & 340 \\
Session Duration (min) & 7.4 & 16.1 & 16.6 \\
Adjustments/min & 33.6 & 32.0 & 20.5 \\
\hline
\end{tabular}
\end{table}

\subsection{FORCE Technology Sound Wheel Mapping}

The classification framework was mapped to the FORCE Technology Sound Wheel attributes to provide a standardized perceptual vocabulary for each sensory profile. Table \ref{tab:sound_wheel} summarizes the recommended audio characteristics for each group.

\begin{table}[htbp]
\centering
\caption{Sound Wheel Attribute Mapping by Sensory Profile}
\label{tab:sound_wheel}
\small
\begin{tabular}{lccc}
\hline
\textbf{Attribute} & \textbf{Hypersensitive} & \textbf{Typical} & \textbf{Hyposensitive} \\
\hline
\multicolumn{4}{l}{\textit{Loudness Category}} \\
Loudness & Soft (30--40 dB) & Medium (50--60 dB) & Loud (65--75+ dB) \\
\hline
\multicolumn{4}{l}{\textit{Dynamics Category}} \\
Attack & Imprecise/Smoothed & Moderate & Precise \\
Bass Precision & Moderate & Moderate & Precise \\
Punch & Weak & Moderate & Strong \\
Powerful & Reduced & Maintained & Maintained \\
\hline
\multicolumn{4}{l}{\textit{Timbre Category}} \\
Treble Strength & A little & Neutral & A lot \\
Brilliance & Low & Moderate & High \\
Tinny & A little & Neutral & A lot \\
Midrange Strength & A little & Neutral & A lot \\
Bass Strength & Neutral & Neutral & A lot \\
Bass Depth & A little & Neutral & A lot \\
Timbral Balance & Dark/Warm & Neutral & Bright \\
Full & Low degree & Moderate & High degree \\
Homogeneous & Coherent & Coherent & Coherent \\
\hline
\multicolumn{4}{l}{\textit{Spatial Category}} \\
Depth & Shallow & Moderate & Deep \\
Width & Small & Moderate & Large \\
Envelopment & Small & Moderate & Large \\
Spatial Balance & Symmetric & Symmetric & Symmetric \\
Distance & Distant & Indistinct & Near \\
Localisability & Imprecise & Moderate & Precise \\
Reverberance & Dry & Moderate & Wet \\
Clarity & Clear (filtered) & Clear & Clear \\
\hline
\multicolumn{4}{l}{\textit{Transparency}} \\
Presence & A little & Moderate & A lot \\
Clean & Distinct & Distinct & Distinct \\
Detail & Simple & Moderate & Rich \\
Natural & A lot & A lot & Neutral \\
\hline
\multicolumn{4}{l}{\textit{Artefacts Category}} \\
Distortion & A little & A little & Moderate \\
Compression & A lot & Neutral & A little \\
\hline
\end{tabular}
\end{table}

\subsection{Key Findings}

\begin{enumerate}
    \item \textbf{Volume Preferences}: Hypersensitive users maintained lower average volumes (51.6\%) compared to hyposensitive users (59.3\%), consistent with established literature on auditory hypersensitivity thresholds \cite{kanakri2017noise}.

    \item \textbf{Muting Behavior}: Hyposensitive participants exhibited higher muting frequency (8.1 events) despite preferring louder volumes, suggesting a pattern of sensory seeking followed by regulation---characteristic of sensory modulation fluctuations.

    \item \textbf{Adjustment Frequency}: Hypersensitive participants showed higher adjustment rates per minute (33.6), indicating active sensory regulation behavior, while hyposensitive participants demonstrated lower rates (20.5), suggesting satisfaction with more intense stimulation levels once achieved.

    \item \textbf{Questionnaire-Behavior Alignment}: Participants who selected ``warm/soft'' sound textures (Option 1) showed 67\% correspondence with hypersensitive classification, while those selecting ``bright/sharp'' (Option 3) aligned 73\% with hyposensitive profiles.

    \item \textbf{Reverb and Delay Preferences}: 91\% of hyposensitive participants selected rich reverb (Option 3), compared to 0\% of hypersensitive participants, validating the theoretical framework that hyposensitive users seek enhanced spatial audio for engagement.
\end{enumerate}

\subsection{Sound Wheel Implementation Recommendations}

Based on the mapping results, the following audio processing recommendations emerge for adaptive VR systems:

\textbf{For Hypersensitive Users:}
\begin{itemize}
    \item Apply low-pass filtering to reduce treble strength (cutoff: 4--6 kHz)
    \item Minimize reverb decay time ($<$ 0.3s)
    \item Apply dynamic compression to limit sudden volume spikes
    \item Favor warm timbral balance with reduced high-frequency content
    \item Maintain shallow spatial depth to reduce auditory complexity
\end{itemize}

\textbf{For Hyposensitive Users:}
\begin{itemize}
    \item Enhance treble and brilliance through parametric EQ boost
    \item Increase reverb decay time (1.0--2.0s) for spatial richness
    \item Enable full dynamic range without compression
    \item Favor bright timbral balance with emphasized high-frequency detail
    \item Maximize spatial depth and envelopment for immersive engagement
\end{itemize}

\textbf{For Typical Users:}
\begin{itemize}
    \item Apply neutral timbral balance
    \item Moderate reverb settings (decay: 0.5--1.0s)
    \item Balanced loudness at comfortable listening levels (50--60 dB SPL)
    \item Maintain natural sound reproduction without heavy processing
\end{itemize}

These mappings provide a scientifically grounded basis for real-time adaptive audio in VR environments, enabling personalized sensory experiences that align with individual neuroperceptual profiles.
