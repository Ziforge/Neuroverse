\subsection{Sampling Limitations and Generalizability}

\subsubsection{Sample Characteristics}

Participants were recruited from a Sound and Music Computing graduate program (N=18), representing a convenience sample with specific demographic and cognitive characteristics. This population is not representative of the general public, and caution must be exercised when generalizing findings.

\subsubsection{Self-Selection Bias in Technical-Auditory Domains}

The observed prevalence of neurodivergent sensory patterns (72.3\%) substantially exceeds general population estimates:

\begin{itemize}
    \item Autism Spectrum Disorder: 2.78\% (Maenner et al., 2023)
    \item Attention-Deficit/Hyperactivity Disorder: 9.4\% in children, 4.4\% in adults (Danielson et al., 2018)
    \item Combined neurodevelopmental conditions: approximately 12-15\%
\end{itemize}

This 5-7$\times$ elevation is consistent with documented self-selection bias in STEM and technical domains. Baron-Cohen et al. (2001) demonstrated that students in mathematics and physical sciences exhibited significantly higher autistic trait scores compared to humanities students. Ruzich et al. (2015) conducted a meta-analysis confirming elevated Autism Quotient scores in science, technology, engineering, and mathematics fields.

\subsubsection{Theoretical Basis for Domain Attraction}

Several mechanisms explain the disproportionate representation of neurodivergent individuals in sound and music computing:

\paragraph{Sensory Sensitivity as Professional Asset}

Robertson and Simmons (2013) documented that autistic individuals frequently exhibit auditory hypersensitivity, enabling detection of subtle acoustic features. In audio engineering and music production contexts, this heightened sensitivity constitutes a professional advantage rather than impairment. Individuals with superior auditory discrimination are more likely to:

\begin{enumerate}
    \item Pursue careers leveraging this capability
    \item Excel in technical audio domains
    \item Self-select into graduate programs emphasizing auditory processing
\end{enumerate}

\paragraph{Cognitive Profile Alignment}

The Sound and Music Computing discipline requires:

\begin{itemize}
    \item Systematic/analytical thinking (digital signal processing)
    \item Pattern recognition (spectral analysis, audio classification)
    \item Attention to detail (mixing, mastering, acoustic measurement)
    \item Technical proficiency (programming, mathematics)
\end{itemize}

These cognitive demands align with autistic cognitive strengths, particularly the systemizing quotient (Baron-Cohen et al., 2003). Consequently, individuals with autistic traits may be disproportionately successful in and attracted to such programs.

\paragraph{Sensation Seeking and Creative Domains}

For ADHD, Panagiotidi et al. (2018) demonstrated associations between ADHD symptoms and atypical auditory processing. The sensation-seeking subtype may be particularly attracted to music production environments offering high stimulation, novelty, and creative expression. White and Shah (2011) found positive associations between ADHD and creative achievement, supporting attraction to creative technical fields.

\subsubsection{Implications for Result Interpretation}

\paragraph{Valid Findings (Sample-Specific)}

The behavioral patterns identified---hypersensitive avoidance, typical balanced response, and hyposensitive sensation seeking---represent genuine phenotypes within this population. The classification accuracy (98.4\%), effect sizes (Cohen's d = 1.0--4.3), and statistical significance (p $<$ 0.001) are valid for the sampled population.

\paragraph{Invalid Generalizations}

The following claims cannot be supported:

\begin{enumerate}
    \item ``50\% of the general population exhibits hypersensitive auditory processing'' --- This prevalence is sample-specific and reflects domain attraction bias
    \item ``Most people with technical audio interests are neurodivergent'' --- Without random sampling, causation and true prevalence cannot be established
    \item ``These classification thresholds apply universally'' --- Thresholds may require recalibration for populations with different neurodivergent base rates
\end{enumerate}

\subsubsection{Recommended Sampling Strategy for Validation}

To establish generalizable findings, stratified sampling is required:

\begin{table}[htbp]
\centering
\caption{Proposed Validation Study Sampling Strategy}
\label{tab:sampling}
\begin{tabular}{lcc}
\hline
\textbf{Population} & \textbf{Target N} & \textbf{Recruitment Source} \\
\hline
Clinically diagnosed ASD & 30 & Autism clinics, support organizations \\
Clinically diagnosed ADHD & 30 & ADHD clinics, psychiatric services \\
Neurotypical controls & 30 & General population, age/gender matched \\
General population (unscreened) & 45 & Community recruitment \\
\hline
\textbf{Total} & \textbf{135} & \\
\hline
\end{tabular}
\end{table}

This design would enable:

\begin{itemize}
    \item Sensitivity/specificity calculation for diagnostic classification
    \item Comparison of behavioral patterns across clinical groups
    \item Estimation of true population-level thresholds
    \item Assessment of ecological validity beyond academic samples
\end{itemize}

\subsubsection{Conclusion on Generalizability}

The current findings demonstrate proof-of-concept that VR behavioral interaction patterns distinguish sensory processing profiles with high accuracy. However, the convenience sample from Sound and Music Computing limits external validity. The elevated neurodivergent prevalence (72.3\% vs. 12-15\% general population) reflects documented self-selection bias in technical-auditory domains (Baron-Cohen et al., 2001; Ruzich et al., 2015), not population-representative prevalence.

Validation with stratified clinical samples is essential before:

\begin{enumerate}
    \item Deploying classification thresholds clinically
    \item Making population-level prevalence claims
    \item Recommending VR behavioral assessment as diagnostic tool
\end{enumerate}

The behavioral phenotypes identified are scientifically valid; their prevalence in the general population remains to be determined through representative sampling.
